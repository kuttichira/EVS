\documentclass[]{article}
% Choose a conveniently small page size
\usepackage[paperheight=16cm,paperwidth=12cm,textwidth=8cm]{geometry}
\usepackage{parskip,enumitem}
\usepackage{tasks}
\settasks{
	style=enumerate,
	label-width={22pt},
	item-indent={22pt+.3333em},
	label-align=right,
	%   debug=true % useful for fine-tuning or debugging
}
%\usepackage{indent}
%opening
\title{Tribal People of India and their Rights.}
\author{Irene Ann Kuttichira}

\begin{document}

\maketitle

\begin{abstract}
\parskip=12pt
% Set the value of \parindent to 10pt
\setlength{\parindent}{15pt}
India is the seventh largest country in the world.  It has a diverse population of tribes, estimated to be 100 million tribal in India across 18 states.  And each tribe have their own unique culture and nature: each tribe should be treated for their unique characteristic. The strange part of the distinct tribes, cannot have a uniform code for all tribes.   Most tribes are confined to different forest lands of India.  As the economic growth of India is on the upward trend, many of the forest lands have been converted into housing complexes to allow expansion of many economically advancing cities of India.  These unwarranted expansion of cities encroaching into the forest lands that were subsistence for the tribal people.  The indigenous people were dependent on the forest vegetation for their dietary and medicinal needs.\newline
\vspace{3ex}\\   %\setlength{\parindent}{15pt}
\hspace{15pt}The republic of India has promulgated many laws in order to protect the tribal needs in India, after protracted years of discussion, ensuring that the tribal people will have equal rights with that of the general population.  The new rules were brought under the \textit{Recognition of Forest Act}, \textbf{2006} as \textbf{THE SCHEDULED TRIBES AND OTHER TRADITIONAL FOREST DWELLERS } which recognize and vest the forest rights and occupations in forest land for forest dwelling of Scheduled Tribes and other forest dwellers who have dwelled in the forest for generations, whose rights were not recorded by the British, and government of India continued with the British laws for forest dwellers and Scheduled Tribes without any amendment. 
\end{abstract}

\section{Tribals - Who are they}
The International Labour Organization (ILO) as been engaged with the indigenour and tribal people's issues since 1929\footnote{\href{https://www.ilo.org/global/topics/indigenous-tribal/lang--en/index.htm}{ - Indigenous and tribal peoples}}.
\vspace{2ex}\\
\footnote{\href{https://www.iasexpress.net/tribal-rights-in-india-constitutional-legal/#&gid=1&pid=1}{ - Notes and Mindmaps on Tribals by IAS Express}}A social group who are not part of the main stream population of a nation before the nation was formed or outside of the community.
They are a group of distinct people, who are dependent on their land for their livelihood, who are self sufficient and not integrated into the mainstream Society.\vspace{2ex}\\   %\setlength{\parindent}{15pt}
\hspace{15pt}They are living in isolated society and are culturally different and form a distinct groups.  They are oldest ethnological section of the population.\vspace{2ex}\\   %\setlength{\parindent}{15pt}
\hspace{15pt}They use primitive methods for occupations such as hunting, gathering of minor forest products.  They have no access for education and predominantly economically backward.  They have their own dialect without any written script.\vspace{3ex}\\   %\setlength{\parindent}{15pt}
\hspace{15pt}
As per the Indian Forest Rights Act, 2006, the tribal people are classified as following:
\begin{tasks}[style=multiline,label-width=20pt]
	\task Based on Location in India\ common forest land within the traditional or customary boundaries of the village or seasonal use of landscape in the case of pastoral communities.
	\task Critical wildlife habitat\ areas of National Parks and Sanctuaries reserved for or the purposes of wildlife conservation.
	\task Forest dwelling Scheduled Tribes\ members or community of the Scheduled Tribes who primarily reside in and who depend on the forests or forest lands for \textit{bona fide} livelihood needs and includes the Scheduled Tribe pastoralist communities.
\end{tasks} 
\end{document}
